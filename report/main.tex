\documentclass[10pt]{report}
\setcounter{tocdepth}{3}
\setcounter{secnumdepth}{3}
\usepackage[french]{babel}
\usepackage[a4paper, left=30mm,right=20mm,top=25mm,bottom=25mm]{geometry}
\usepackage{graphicx} % Required for inserting images
\usepackage[T1]{fontenc}

\setlength{\parskip}{1ex plus 0.5ex minus 0.2ex}
\newcommand{\hsp}{\hspace{20pt}}
\newcommand{\HRule}{\rule{\linewidth}{0.5mm}}
\renewcommand\thesection{\arabic{section}}

\begin{document}

\begin{titlepage}
  %\begin{sffamily}
  \begin{center}
    \includegraphics[height=25mm]{gfx/logo-UMLP.png}~\\[3cm]

    \textsc{\LARGE UFR Sciences et technique}\\
    \textsc{\Large Université Marie et Louis PASTEUR}\\
    \textsc{\Large \textbf{Projet}}\\
    \textsc{\Large L3 Informatique}\\[2.5cm]

    \HRule \\[0.4cm]
    { \huge \bfseries Jeu de plate-formes avec génération procédurale\\[0.4cm] } % C'est le titre donné par jube, j'y suis pour rien

    \HRule \\[0.6cm]\renewcommand\thesection{\arabic{section}}
    \textsc{\Large \textbf{Rapport de projet}}\\[1.5cm]
    \begin{minipage}{\textwidth}
      \begin{center}\LARGE
        Kilian \textsc{Jelic} \\
        Laura \textsc{Jacqueson} \\
        Théo \textsc{Pariney} \\
      \end{center}
    \end{minipage}
    \vfill
    % Bottom of the page
    {\large \ avril 2025}

  \end{center}
  %\end{sffamily}
\end{titlepage}

\normalsize
\pagenumbering{arabic}
\tableofcontents
\pagebreak
\listoffigures
\pagebreak

\section{Introduction}

Dans le cadre de notre projet semestriel en L3 Informatique à l'Université Marie et Louis Pasteur, nous avons travaillé sur le développement d’un jeu de plate-formes avec génération procédurale en C++. 

Les jeux de plate-formes sont un genre qui repose sur le contrôle d'un personnage avec des mécaniques comme les sauts et les obstacles, l’objectif étant généralement de rejoindre une sortie pour terminer le niveau. L’ajout de la génération procédurale introduit une caractéristique supplémentaire : plutôt que de concevoir chaque niveau à la main, un algorithme est chargé de créer les niveaux du jeu, permettant ainsi une expérience unique à chaque partie.

L'objectif principal de ce projet était de concevoir un jeu de plate-formes dont les niveaux seraient générés de manière procédurale, offrant ainsi une rejouabilité infinie. En parallèle, nous avons dû concevoir un jeu intégrant plusieurs mécaniques de jeu, incluant des déplacements, des sauts, un dash, ainsi qu’un objectif centré sur la récolte d’objets, tout en assurant la gestion des collisions et la physique du jeu.

Dans ce rapport, nous commencerons par une présentation générale du jeu, en exposant les objectifs du joueur et les scènes du jeu. Ensuite, nous détaillerons la conception des blocs et entités qui composent le monde et leur gestion dynamique. Par la suite, nous aborderons la génération procédurale, en expliquant la méthode utilisée ainsi que les détails d'implémentation. Nous discuterons ensuite des contrôles du personnage et des différentes mécaniques de jeu, telles que le saut, le dash, etc. La partie suivante sera consacrée à la physique, notamment la gestion des collisions et des propriétés physiques comme la résistance de l'air. Enfin, nous reviendrons sur l’organisation du travail, la répartition des tâches et les outils utilisés pour mener à bien ce projet. Le rapport se conclura par un bilan du projet et les améliorations possibles.

\pagebreak



\section{Présentation du jeu}
\subsection{Présentation générale}

Ce projet consiste à développer un jeu de plate-forme en 2D intégrant un système de génération procédurale de niveau. L’objectif est d’offrir une expérience où chaque partie est unique : au lieu d’avoir des niveaux prédéfinis, ceux-ci sont générés aléatoirement à chaque nouvellepartie. Contrairement aux jeux de plate-forme classiques où l'on peut apprendre les niveaux par cœur, ici, aucun parcours ne peut être rejoué à l’identique. Cette approche permet non seulement une rejouabilité infinie, mais aussi un défi constant, obligeant le joueur à s’adapter à chaque nouvelle configuration de niveau.

Le joueur incarne un petit robot, dont la mission principale est de récolter un maximum d’écrous avant d’atteindre la sortie du niveau. Ces écrous sont disposés aléatoirement dans l’environnement, incitant le joueur à explorer avant de pouvoir terminer la partie.

Pour évoluer dans le monde, le joueur dispose de plusieurs mécaniques lui permettant de se déplacer librement et d’interagir avec son environnement :
\begin{itemize}
  \item \textbf{déplacement} : le personnage peut se déplacer latéralement, à gauche et à droite, pour parcourir le niveau.
  \item \textbf{saut} : le personnage peut sauter pour franchir des obstacles ou atteindre des plateformes situées en hauteur.
  \item \textbf{double saut} : après un premier saut, lorsqu'il est encore en l’air, le personnage peut effectuer un second saut lui permettant d’atteindre des zones plus élevées.
  \item \textbf{dash} : le personnage dispose également d’un dash, une impulsion rapide dans une direction (gauche ou droite), utile pour traverser de grands espaces vides ou esquiver des obstacles.\\
\end{itemize}

Ces mécaniques offrent une grande liberté de mouvement, permettant aux joueurs d’adopter différentes stratégies. Certains privilégieront une approche prudente, optimisant chaque saut pour éviter la mort, tandis que d’autres tenteront des enchaînements rapides et fluides pour terminer le niveau efficacement.


Comme mentionné précédemment, tous les niveaux sont générés de manière procédurale, cela signifie qu'à chaque lancement de partie c'est un algorithme qui se charge de créer un environnement en plaçant des blocs, plateformes, échelles, écrous, piques et une sortie. Cette méthode permet de générer de manière presque infinie des mondes différents. 

%ajout d'une petite présentation de l'algo de génération%


\subsection{Les objectifs du joueur}
\subsubsection{Les écrous}

\subsubsection{La sortie}

\subsection{Les scènes}


\section{Les blocks et entités du monde}
\subsection{La définition dynamique des blocks}
% Fichier XML et récupération dans une classe dédiée
\subsection{Le stockage des blocks}
% Récupération dans le blockmanager utilisé par le monde

\section{La génération procédurale}
% Je sais pas trop comment ça se passe, faudrait le remplir
% Faudra penser à préciser qu'on a commencé par faire un monde "test" avant d'arriver à notre générateur actuel
% Et aussi à donner les étapes/difficultés avant d'en arriver à un truc qui marche correctement

\section{Le personnage}
\subsection{Les contrôles de base}
\subsection{Le saut}
\subsubsection{Le saut modulable}
%ça fait partie des choix qu'on a fait puis abandonné, c'est important à mentionner
\subsubsection{Le double saut}
\subsection{Le dash}
%Faudra aussi mentionner qu'on a eu du mal à trouver des contrôles convenables
\subsection{Le score et les vies}

\section{La physique}
\subsection{La gestion des collisions}
\subsubsection{Les propriétés des blocks}
\subsection{Calcul de la résistance de l'air}

\section{Notre organisation}
\subsection{La définition des tâches}
\subsubsection{La répartition du travail}
\subsection{Les outils utilisés}
% ça serait important de mentionner qu'on a mis un peu de temps à avoir un trello
% je pense aussi que l'ordre dans lequel on a mis les outils en place peut être interessant à rajouter
\subsubsection{Discord}
\subsubsection{Google Docs} %Pour les idées à la base
\subsubsection{Trello}
\subsubsection{Git et Github}
\subsubsection{IDE}
% Cette section là elle est marrante parse que personne utilise le même
\newpage

\section{Conclusion}
% On a quand même une conclusion ici avant les autres points
\subsection{Points d'amélioration}
\subsubsection{Ce qu'on aurait voulu ajouter}
%c.f le gdoc
\subsubsection{Wave Function Collapse}
\subsubsection{Pathfinding amélioré}
\subsubsection{La physique} % ;-;


\end{document}
