\documentclass{beamer}

\usepackage[french]{babel}
\usepackage[T1]{fontenc}
\usepackage[export]{adjustbox}
\usepackage{graphicx}
\usepackage{blkarray} % Required for inserting images

\newcommand{\nologo}{\setbeamertemplate{logo}{}}

\mode<presentation>{
  \useinnertheme{rectangles}
  \usecolortheme[rgb={0.3,0.5,0.8}]{structure}
  \usecolortheme{orchid}
  \usecolortheme{whale}
  \useoutertheme{infolines}
  \setbeamercovered{transparent}
}

\graphicspath{ {./images/} }

\title{Projet semestriel}
\subtitle{Jeu de plate-formes avec génération procédurale}
\author[T.Pariney L.Jaqueson K.Jelic]{Théo Pariney \newline Laura Jacqueson \newline Kilian Jelic\\\footnotesize Tuteur : Julien Bernard}
\institute[]{Université Marie et Louis Pasteur \\ \vspace{0.25cm} Licence 3 Informatique, 2024--2025}
\date{26 mars 2025}
\logo{\includegraphics[width=0.25\linewidth]{images/logo-UMLP}}

\begin{document}

\begin{frame}
    \titlepage
\end{frame}

{\nologo

\begin{frame}{Plan}
    \tableofcontents
\end{frame}

\begin{frame}{Introduction}
    \begin{block}{Contexte}
        \begin{itemize}
            \item[\bullet] Jeu de platformes en 2D
            \item[\bullet] Génération procédurale
            \item[\bullet] Réalisé en C++
            \item[\bullet] Utilise Gamedev Framework
        \end{itemize}
    \end{block}
    \begin{block}{Objectifs}
       \begin{itemize}
            \item[\bullet] Programmer un jeu en C++
            \item[\bullet] Réaliser les textures
            \item[\bullet] Se familiariser avec la bibliothèque GF
        \end{itemize}
    \end{block}
\end{frame}

\section{Présentation générale}
\begin{frame}{Présentation du projet}

\end{frame}

\begin{frame}{Objectifs du jeu}

\end{frame}

\begin{frame}{Construction du jeu en scènes}

\end{frame}

\section{Blocs et textures}
\begin{frame}{Stockage dynamique des blocs}
    \begin{columns}
        \column{0.6\textwidth}
            \begin{figure}
                \centering
                \includegraphics[width=1.0\textwidth]{images/XMLfile}
                \caption{Stockage d'un bloc au format XML}
            \end{figure}
        \column{0.4\textwidth}
            \begin{block}{Définition des blocs}
                \begin{itemize}
                    \item[\bullet] Définition des blocs au format XML
                    \item[\bullet] Lecture du fichier et stockage dans une classe dédiée
                    \item[\bullet] Accès depuis les autre classes via une méthode statique
                \end{itemize}
            \end{block}
    \end{columns}
\end{frame}

\begin{frame}{Connectivité des textures}
    \begin{columns}
        \column{0.4\textwidth}
            \begin{block}{Textures connectées}
                \begin{itemize}
                    \item[\bullet] Textures des blocs connectées aux blocs du même type
                    \item[\bullet] élément \emph{connected} du fichier XML
                    \item[\bullet] Textures au format de feuilles de sprite
                    \item[\bullet] Calcul basé sur les blocs autour
                \end{itemize}
            \end{block}
        \column{0.6\textwidth}
            \begin{figure}
                \centering
                \includegraphics[width=0.8\textwidth]{images/connected_textures_offset_computing}
                \caption{Calcul de la texture connectée du bloc}
            \end{figure}
    \end{columns}
\end{frame}

\section{Présentation du personnage}
\begin{frame}{Personnage (1/4)}

\end{frame}

\begin{frame}{Personnage (2/4)}

\end{frame}

\begin{frame}{Personnage (3/4)}

\end{frame}

\begin{frame}{Personnage (4/4)}

\end{frame}

\section{La génération procédurale}
\begin{frame}{L'al\'eatoire}
    \begin{block}{Graine de génération}
        \begin{itemize}
            \item[\bullet] Choisie aléatoirement
            \item[\bullet] Peut être choisie manuellement
            \item[\bullet] Permet de reproduire les problèmes
        \end{itemize}
    \end{block}
\end{frame}

\begin{frame}{Génération des salles/des murs}
    \begin{columns}
        \column{0.5\textwidth}
        \begin{block}{Génération des salles}
            \begin{itemize}
                \item[\bullet] Génère une suite de salles connectées entre elles
                \item[\bullet] Éviter que les salles se chevauchent
                \item[\bullet] Retourne en arrière si aucun chemin n'est possible
            \end{itemize}
        \end{block}
        \column{0.4\textwidth}
        \begin{figure}
            \centering
            \includegraphics[height=0.2\textheight]{images/room_placement}
            \caption{Placement d'une salle au dessus d'une autre}
        \end{figure}
        \begin{figure}
            \centering
            \includegraphics[height=0.2\textheight]{images/filling_the_world}
            \caption{Génération des murs}
        \end{figure}
    \end{columns}
\end{frame}

\begin{frame}{Génération du chemin}
    \begin{columns}
        \column{0.5\textwidth}
        \begin{block}{Génération d'un chemin}
            \begin{itemize}
                \item[\bullet] Génère un chemin du début à la fin
                \item[\bullet] S'assurer que le niveau est possible
            \end{itemize}
        \end{block}
        \column{0.4\textwidth}
        \begin{figure}
            \centering
            \includegraphics[width=1.0\textwidth]{images/two_ways_to_connect}
            \caption{Deux moyens de relier deux points du chemin}
        \end{figure}
    \end{columns}
\end{frame}

\begin{frame}{Les fausses platformes}
    \begin{columns}
        \column{0.5\textwidth}
        \begin{block}{Pourquoi?}
            \begin{itemize}
                \item[\bullet] Cache le vrai chemin
                \item[\bullet] Exploration du niveau
            \end{itemize}
        \end{block}
        \column{0.4\textwidth}
        \begin{figure}
            \centering
            \includegraphics[width=1.0\textwidth]{images/fake_platforms}
            \caption{Une salle contenant de fausses plateformes}
        \end{figure}
    \end{columns}
\end{frame}

\begin{frame}{Les blocs spéciaux}
    \begin{columns}
        \column{0.5\textwidth}
        \begin{block}{Blocs de glace et de slime}
            \begin{itemize}
                \item[\bullet] Ajoute de la variété
                \item[\bullet] Généré à partir d'un bruit de Perlin
            \end{itemize}
        \end{block}
        \column{0.4\textwidth}
        \begin{figure}
            \centering
            \includegraphics[height=0.2\textheight]{images/perlin_noise_example}
            \caption{Données brutes d'un bruit de Perlin}
        \end{figure}
        \begin{figure}
            \centering
            \includegraphics[height=0.2\textheight]{images/perlin_noise_in_level}
            \caption{Des blocs de glace et de slime dans un niveau}
        \end{figure}
    \end{columns}
\end{frame}

\begin{frame}{Les pièges}
    \begin{columns}
        \column{0.5\textwidth}
        \begin{block}{Génération de pics}
            \begin{itemize}
                \item[\bullet] Dans les salles dangereuses
                \item[\bullet] Au bout de lignes de blocs de glace
            \end{itemize}
        \end{block}
    \column{0.4\textwidth}
        \begin{figure}
            \centering
            \includegraphics[width=1.0\textwidth]{images/ice_spike_trap}
            \caption{Deux situations où des pics peuvent se générer}
        \end{figure}
    \end{columns}
\end{frame}

\begin{frame}{Point d'apparition}
    \begin{columns}
        \column{0.5\textwidth}
        \begin{block}{Point d'apparition du joueur}
            \begin{itemize}
                \item[\bullet] Début du niveau
                \item[\bullet] Dans un endroit sûr
            \end{itemize}
        \end{block}
        \column{0.4\textwidth}
        \begin{figure}
            \centering
            \includegraphics[width=1.0\textwidth]{images/valid_spawn_locations}
            \caption{Points d'apparition valides dans un niveau}
        \end{figure}
    \end{columns}
\end{frame}

\begin{frame}{Monde de test}
    \begin{columns}
        \column{0.5\textwidth}
        \begin{block}{Générateur de monde de test}
            \begin{itemize}
                \item[\bullet] Implémentation alternative d'un générateur de monde
                \item[\bullet] Permet de tester les fonctionnalités
            \end{itemize}
        \end{block}
        \column{0.4\textwidth}
        \begin{figure}
            \centering
            \includegraphics[width=1.0\textwidth]{images/test_world}
            \caption{Capture d'écran du monde de test}
        \end{figure}
    \end{columns}
\end{frame}

\section{La physique}
\begin{frame}{Gestion des collisions}
    Calcul de la vitesse par rapport à la normale :
    \[ V_{n} = - V_{joueur} \cdot n \]
    Calcul de la magnitude de l'impulsion à appliquer :
    \[ C_{i} = (1 + e) \times V_{n} \]
    Calcul du vecteur résultat :
    \[V_{res} = C_{i} \times n \]
    Application d'une correction (pour les erreurs de flottant) :
    \[V_{corr} = -max(d-0.1,0) \times C_{corr} \times n\]
\end{frame}

\begin{frame}{Platformes directionnelles}
    \begin{columns}
        \column{0.4\textwidth}
            \begin{block}{Platformes directionnelles}
                \begin{itemize}
                    \item[•] Utilisent une hitbox plus petite, dirigée
                    \item[•] élément \emph{direction} du fichier XML
                    \item[•] Le joueur utilise aussi une hitbox plus petite
                \end{itemize}
            \end{block}
        \column{0.6\textwidth}
            \begin{figure}
                \centering
                \includegraphics[width=0.5\textwidth]{images/DirectionnalHB.png}
                \caption{Hitbox directionnelle comparée à la hitbox classique}
            \end{figure}
    \end{columns}
\end{frame}

\begin{frame}{Friction}

\end{frame}

\begin{frame}{Stockage des données}

\end{frame}

\section{Conclusion}
\begin{frame}{Conclusion}

\end{frame}

\begin{frame}{Ce qu'on aurait voulu ajouter}

\end{frame}

\begin{frame}{Wave function collapse}
    \begin{columns}
        \column{0.5\textwidth}
        \begin{block}{Algorithme Wave Function Collapse}
            \begin{itemize}
                \item[\bullet] Arrange un ensemble de patternes donnés
                \item[\bullet] Pourrait être utilisé avec des morceaux de niveau
            \end{itemize}
        \end{block}
        \column{0.4\textwidth}
        \begin{figure}
            \centering
            \includegraphics[width=1.0\textwidth]{images/wfc-examples}
            \caption{Des exemples d'application du Wave Function Collapse}
        \end{figure}
    \end{columns}
\end{frame}

\begin{frame}{Amélioration du pathfinding}
    \begin{block}{Meilleur pathfinding}
        \begin{itemize}
            \item[\bullet] Véritable algorithme de pathfinding
            \item[\bullet] Fonctionne 100\% du temps
            \item[\bullet] Prend en compte les contraintes
            \item[\bullet] Trajectoire rectiligne
        \end{itemize}
    \end{block}
\end{frame}

}


\end{document}
