\documentclass{beamer}

\usepackage[french]{babel}
\usepackage[T1]{fontenc}
\usepackage[export]{adjustbox}
\usepackage{graphicx} % Required for inserting images

\newcommand{\nologo}{\setbeamertemplate{logo}{}}

\mode<presentation>{
  \useinnertheme{rectangles}
  \usecolortheme[rgb={0.3,0.5,0.8}]{structure}
  \usecolortheme{orchid}
  \usecolortheme{whale}
  \useoutertheme{infolines}
  \setbeamercovered{transparent}
}

\graphicspath{ {./images/} }

\title{Projet semestriel}
\subtitle{Jeu de plate-formes avec génération procédurale}
\author[T.Pariney L.Jaqueson K.Jelic]{Théo Pariney \newline Laura Jacqueson \newline Kilian Jelic\\\footnotesize Tuteur : Julien Bernard}
\institute[]{Université Marie et Louis Pasteur \\ \vspace{0.25cm} Licence 3 Informatique, 2024--2025}
\date{26 mars 2025}
\logo{\includegraphics[width=0.25\linewidth]{images/logo-UMLP.png}}

\begin{document}

\begin{frame}
    \titlepage
\end{frame}

{\nologo

\begin{frame}{Plan}
    \tableofcontents
\end{frame}

\begin{frame}{Introduction}
    \begin{block}{Contexte}
        \begin{itemize}
            \item[•] Jeu de platformes en 2D
            \item[•] Génération procédurale
            \item[•] Réalisé en C++
            \item[•] Utilise Gamedev Framework
        \end{itemize}
    \end{block}
    \begin{block}{Objectifs}
       \begin{itemize}
            \item[•] Programmer un jeu en C++
            \item[•] Réaliser les textures
            \item[•] Se familiariser avec la bibliothèque GF
        \end{itemize}
    \end{block}
\end{frame}

\section{Présnetation générale}
\begin{frame}{Présentation du projet}

\end{frame}

\begin{frame}{Objectifs du jeu}

\end{frame}

\begin{frame}{Construction du jeu en scènes}

\end{frame}

\section{Blocs et textures}
\begin{frame}{Stockage dynamique des blocs}

\end{frame}

\begin{frame}{Connectivité des textures}

\end{frame}

\section{Présentation du personnage}
\begin{frame}{Personnage (1/4)}

\end{frame}

\begin{frame}{Personnage (2/4)}

\end{frame}

\begin{frame}{Personnage (3/4)}

\end{frame}

\begin{frame}{Personnage (4/4)}

\end{frame}

\section{La génération procédurale}
\begin{frame}{L'aléatoire}

\end{frame}

\begin{frame}{Génération des salles/des murs}

\end{frame}

\begin{frame}{Génération du chemin}

\end{frame}

\begin{frame}{Les fausses platformes}

\end{frame}

\begin{frame}{Les blocs spéciaux}

\end{frame}

\begin{frame}{Les pièges}

\end{frame}

\begin{frame}{Point d'apparition}

\end{frame}

\begin{frame}{Monde de test}

\end{frame}

\section{La physique}
\begin{frame}{Gestion des collisions}

\end{frame}

\begin{frame}{Platformes directionnelles}

\end{frame}

\begin{frame}{Friction}

\end{frame}

\begin{frame}{Stockage des données}

\end{frame}

\section{Conclusion}
\begin{frame}{Conclusion}

\end{frame}

\begin{frame}{Ce qu'on aurait voulu ajouter}

\end{frame}

\begin{frame}{Wave function collapse}

\end{frame}

\begin{frame}{Amélioration du pathfinding}

\end{frame}

}


\end{document}
